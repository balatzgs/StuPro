% !TeX spellcheck = de_DE

\chapter{Zusammenfassung und Ausblick}\label{chap:zusfas}
Wärmebildkameras können dazu genutzt werden, die visuelle Wahrnehmung zu verstärken.
Für Feuerwehren sind diese Kameras besonders nützlich, allerdings ist eine Einschätzung von Distanzen und die räumliche Wahrnehmung, mit Wärmebildkameras allein, schwer.
Auch Wärmereflexionen sind ein häufig auftretendes Problem.
Aus diesen Gründen wurde das Projekt \profire gestartet.
Zu Anfang des Projekts \profire festgelegt, dass bis zum 31.3.2016 ein Prototyp, welcher eine Bildfusion einer Wärmebildkamera mit einer Tiefenbildkamera realisiert, entwickelt und abgegeben werden muss.
Neben dieser Fusion wurde eine Halterung entwickelt, welche es erlaubt, beide Kameras als tragbares Gerät, wie eine normale Wärmebildkamera zu nutzen.
Diese Halterung kann allerdings auch an einem Helm befestigt werden und mit einem Headmounted-Display kombiniert, verwendet werden.

Um ein möglichst gutes Bildergebnis zu erreichen, werden erst die Tiefenbilder mit der Methode der Dilatation verbessert.
Nach Kalibrierung der beiden Kameras, wurden beide Bildmodi, durch ein vervierfachen der Auflösung mit anschließender Aufteilung der Pixel für Tiefenbild und Wärmebild, fusioniert.

Darauf wurde der Prototyp mit Mitgliedern einer freiwilligen Feuerwehr evaluiert.
Dabei konnten einige Verbesserungsvorschläge und viel positives Feedback gesammelt und auch nachgewiesen werden.

\section*{Ausblick}
Da allerdings die Feuerwehrleute, welche bereits Erfahrung mit Wärmebildkameras hatten, im Gegensatz zu einer Anfängergruppe, mit keinerlei Tiefen- oder Wärmebilderfahrung, kaum das normale Tiefenbild als hilfreich für die Navigation empfanden, wäre einer der nächsten Schritte, eine neue Nutzergruppe zu erstellen.
Diese sollte dann über einen längeren Zeitraum an alle drei Bildmodi gewöhnt werden.
Die folgende Evaluation wäre aussagekräftiger, als die durchgeführten Schritte.

Zudem muss die Tiefenbildkamera auf lange Sicht, mit einem Sonar oder Radar ausgewechselt werden, da auch der Tiefenbildsensor Probleme mit Spiegelungen hat.
Zusätzlich dazu ist die funktionstüchtigen Distanz stark begrenzt und nahezu untauglich in verrauchten Umgebungen.
