% !TeX spellcheck = de_DE

\chapter{Einleitung}
%In diesem Kapitel steht die Einleitung zu dieser Arbeit.
%Sie soll nur als Beispiel dienen und hat nichts mit dem Buch~~\cite{WSPA} zu tun.
%Nun viel Erfolg bei der Arbeit!
%
%Bei \LaTeX\ werden Absätze durch freie Zeilen angegeben.
%Da die Arbeit über ein Versionskontrollsystem versioniert wird, ist es sinnvoll, pro \emph{Satz} eine neue Zeile im \texttt{.tex}-Dokument anzufangen.
%So kann einfacher ein Vergleich von Versionsständen vorgenommen werden.
Feuerwehrleute können sich im Einsatz nicht immer auf ihre visuelle Wahrnehmung verlassen, unter anderem wegen Rauch ist es in brennenden Gebäuden dunkel.
In solchen Situationen schnell den Brandherd, andere Gefahrenstellen oder Hilfsbedürftige zu erkennen ist wichtig.
Auch in anderen Situationen, wie \zB der Menschenrettung, können Wärmebildkameras, wegen ihrer hohen Reichweite, eine Hilfe sein.
Daher ist es nicht verwunderlich, dass Wärmebildkameras heutzutage in den meisten Feuerwehren anzutreffen sind.

Allerdings haben viele Feuerwehrleute Probleme damit, sich räumlich gut zurecht zu finden und Entfernungen korrekt einschätzen zu können.
Navigation und Orientierung sind damit deutlich eingeschränkt.
Auch Wärmereflexionen stellen häufig ein Problem dar.
Um dem entgegenzuwirken wurde das Projekt \profire gegründet.
\profire versucht die Schwächen von Wärmebildkameras durch das hinzufügen einer Tiefenbildkamera auszugleichen.
Dabei werden die gewonnenen Tiefeninformationen dazu genutzt eine Bildfusion aus beiden Bildern zu erstellen.

Dazu wird ein physischer Prototyp entwickelt, welcher sowohl als tragbare Handheld-Version besteht, als auch in bestehende Feuerwehrausrüstung integriert werden kann.

Als erstes werden in diesem Dokument die zu erbringenden Leistungen des Projektteams beschreiben.
Anschließend folgen die Rahmenbedingungen des Projekts und eine Spezifikation des zu entwickelnden Prototypen.
Drauf folgt eine Beschreibung des erstellten Codes.
Dies ist gefolgt von der Erklärung der verschiedenen gewählten Ansätze zur Bildverbesserung, Dilatation und Kalibrierung.
Danach wird auf die verschiedenen betrachteten Fusionsansätze eingegangen.
Das ganze ist gefolgt von zwei Studienberichten, welche den Prototypen zu unterschiedlichen Lebenszyklen mit unterschiedlich gut geschulten Probanden evaluierten. 
Diese bescheinigen dem Prototypen Erfolge bei der Stärkung der räumlichen Wahrnehmung und zeigten zudem auch eine benutzerfreundliche Handhabung.

\clearpage

\section*{Gliederung}
Die Arbeit ist in folgender Weise gegliedert:
\begin{description}
\item[\cref{chap:angebot} -- \nameref{chap:angebot}] beschreibt die gegebenen Einschränkungen und zu erfüllenden Anforderungen an das Endprodukt.
\item[\cref{chap:projektplan} -- \nameref{chap:projektplan}:] hält Ziele, Zeitplanung und Bedingungen des Projekts fest.
\item[\cref{chap:spezi} -- \nameref{chap:spezi}:] beschreibt die funktionale und nicht-funktionale Anforderungen und die Funktionsweise des Endproduktes.
\item[\cref{chap:entwurf} -- \nameref{chap:entwurf}:] enthält das Grundgerüst und Komponentenbeschreibungen des Programms.
\item[\cref{chap:dilat} -- \nameref{chap:dilat}] erklärt den Dilatationsprozess, durch welchen die Tiefenbilder verbessert wurden.
\item[\cref{chap:calibration} -- \nameref{chap:calibration}] fasst die Vorgänge und Ergebnisse des Kalibrationsprozesses zusammen.
\item[\cref{chap:fusion} -- \nameref{chap:fusion}:] geht auf die explorierten Fusionsmöglichkeiten ein.
\item[\cref{chap:study} -- \nameref{chap:study}:] beschreibt Studienaufbau, Studiendurchführung und Studienauswertung.
\item[\cref{chap:zusfas} -- \nameref{chap:zusfas}] fasst die Ergebnisse des Projekts zusammen und stellt Anknüpfungspunkte vor.
\item[\cref{chap:install} -- \nameref{chap:install}] enthält die Installations- und Startanleitung des Programms.
\end{description}
