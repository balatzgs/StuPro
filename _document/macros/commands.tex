%wird fuer Tabellen benötigt (z.B. >{centering\RBS}p{2.5cm} erzeugt einen zentrierten 2,5cm breiten Absatz in einer Tabelle
\newcommand{\RBS}{\let\\=\tabularnewline}

%% typoraphisch richtige Abkürzungen
\newcommand{\zB}[0]{zum Beispiel\xspace}
\newcommand{\ca}[0]{circa\xspace}
\newcommand{\bzw}[0]{beziehungsweise\xspace}
\newcommand{\usw}[0]{usw.\xspace}
\newcommand{\matr}[0]{Matrikelnummer\xspace}
\newcommand{\email}[0]{E-Mail\xspace}
\newcommand{\emails}[0]{E-Mails\xspace}
\newcommand{\bspw}[0]{beispielsweise\xspace}
\newcommand{\bzgl}[0]{bezüglich\xspace}
\newcommand{\ggf}[0]{gegebenenfalls\xspace}
\newcommand{\setup}[0]{Set-Up\xspace}
\renewcommand{\dh}[0]{das heißt\xspace}
\newcommand{\Dh}[0]{Das heißt\xspace}
\newcommand{\va}[0]{vor allem\xspace}
\newcommand{\ea}[0]{et~al.\xspace}
\newcommand{\meta}[0]{Meta~1\xspace}
\newcommand{\profire}[0]{\enquote{\textbf{ProFire}}\xspace}

%from hmks makros.tex - \indexify
\newcommand{\toindex}[1]{\index{#1}#1}
%
\newcommand{\dotcup}{\ensuremath{\,\mathaccent\cdot\cup\,}} %Tipp aus The Comprehensive LaTeX Symbol List
%
%Anstatt $|x|$ $\abs{x}$ verwenden. Die Betragsstriche skalieren automatisch, falls "x" etwas größer sein sollte...
\newcommand{\abs}[1]{\left\lvert#1\right\rvert}
%
%für Zitate
\newcommand{\citeS}[2]{\cite[S.~#1]{#2}}
\newcommand{\citeSf}[2]{\cite[S.~#1\,f.]{#2}}
\newcommand{\citeSff}[2]{\cite[S.~#1\,ff.]{#2}}
\newcommand{\vgl}{vergleiche\ }
\newcommand{\Vgl}{Vergleiche\ }
%
\newcommand{\commentchar}{\ensuremath{/\mkern-4mu/}}
\algrenewcommand{\algorithmiccomment}[1]{\hfill $\commentchar$ #1}

% Seitengrößen - Gegen Schusterjungen und Hurenkinder...
\newcommand{\largepage}{\enlargethispage{\baselineskip}}
\newcommand{\shortpage}{\enlargethispage{-\baselineskip}}

\newcommand{\goal}[0]{Ziel}
\newcommand{\precondition}[0]{Vorbedingung}
\newcommand{\postcondition}[0]{Nachbedingung}
\newcommand{\postexception}[0]{Nachbedingung im Sonderfall}
\newcommand{\flow}[0]{Regulärer Ablauf}
\newcommand{\exception}[0]{Sonderfall}
\newcommand{\player}[0]{Akteure}

\newcommand{\term}[0]{Begriff, Synonyme}
\newcommand{\intent}[0]{Bedeutung}
\newcommand{\bound}[0]{Abgrenzung}
\newcommand{\validity}[0]{Gültigkeit}
\newcommand{\identifier}[0]{Bezeichnung}
\newcommand{\blur}[0]{Unklarheiten}
\newcommand{\crossref}[0]{Querverweise}